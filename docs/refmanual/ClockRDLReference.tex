\documentclass[11pt, oneside]{article}   	% use "amsart" instead of "article" for AMSLaTeX format
\usepackage{geometry}                		% See geometry.pdf to learn the layout options. There are lots.
\geometry{letterpaper}                   		% ... or a4paper or a5paper or ... 
%\geometry{landscape}                		% Activate for for rotated page geometry
%\usepackage[parfill]{parskip}    		% Activate to begin paragraphs with an empty line rather than an indent
\usepackage{graphicx}				% Use pdf, png, jpg, or eps§ with pdflatex; use eps in DVI mode
								% TeX will automatically convert eps --> pdf in pdflatex		
\usepackage{amssymb}
\usepackage[names]{xcolor}
\usepackage{listings}
\usepackage{amssymb}
\usepackage{algorithmicx}
\usepackage{algorithm}
\usepackage{algpseudocode}
\usepackage{multirow}
\usepackage{paralist}
\usepackage[breaklinks]{hyperref}
\usepackage{xspace}
\usepackage{subfig}

\title{ClockRDL: The Clock Relation Definition Language}
\author{Ciprian Teodorov}
\date{\today}							% Activate to display a given date or no date

\begin{document}
\maketitle
%\section{}
%\subsection{}
\begin{abstract}

This document describes ClockRDL, a textual specification language based on the [MoCCML](http://mocml) formalism.

\end{abstract}

\section{Introduction}
ClockRDL is a constraint specification language based on the Leslie Lamport's `logical clocks` representation of time. It is used for specifying, executing, and analyzing the relations between interesting events in a system of interest.

\subsection{Logical time constraints for system specification}
Unlike traditional specification languages, which focus on describing how a system achieves a given task, ClockRDL specifications describe how the events of interest in a system are related to one another. From this perspective the relation between events, specified with ClockRDL, should be seen as an executable encoding of high-level requirements.

**simple example*

\subsection{Other Approaches}

- Synchronous Languages
- CCSL
- MoCCML
- ClockSystem

\subsection{Applications}
- Concurrency semantics
- Timing constraints in critical systems


\subsection{Features}
- Automaton-based relation specification
- Minimal, clear, and simple textual syntax

\subsection{Design Goals}
- ClockRDL should be a practical relation description language, which favors a clear human readable syntax.
- ClockRDL should have a clear semantics 
- The relations described with ClockRDL should be readable by anyone familiar with state-machines
- The relations should be easy to explain and understand.
- Besides the need for a `specialized clock constraints` execution engine, the implementation should not be overly complex
- ClockRDL should be simple enough to enable experimentation with new features.

\subsection{Audience}
\textcolor{red}{
This reference manual is intended to software professionals and language implementors. Through this manual we attempt to keep the descriptions simple and precise while providing numerous examples.
Nonetheless, this manual is not a tutorial of ClockRDL nor is it a guide for using ClockRDL in various application scenarios.
}

\section{Language Reference}
\subsection{Lexical Structure}
\subsection{Values}
\subsection{Types}
\subsection{Statements and Expressions}
\subsection{Declarations}
\subsection{Relations}
\subsection{Libraries}
\section{Semantics}

\subsection{Reachability Analysis}
Formally, a ClockRDL relation R is a 4-tuple $\langle S, I, C, V, Pre, Eff, \rightarrow \rangle$ with:
\begin{itemize}
\itemsep0em
\item[-] $S$ is a set of states;
\item[-] $I$ is the initial state: $I \in S$;
\item[-] $C$ is a set of clocks, known as the alphabet of the relation;
\item[-] $V$ is a set of variables assuming values from finite domains;
\item[-] $Pre$ is a set of preconditions;
\item[-] $Eff$ is a set of effects (actions);
\item[-] $\rightarrow \subseteq S \times clks \times Pre \cup \emptyset \times Eff \cup \emptyset \times S$ is a transition relation, where $clks \subseteq C \cup \emptyset$
%\footnote{Sometimes $\rightarrow$ is considered a total relation and Act is extended with $\{\tau\}$ -- an internal hidden action}, we will also use $\mathbb{T} \subseteq S \times Act \times S$ when referring to the set of transitions of the LTS;
%\item[-] $L$ is a function that associates a state-vector to each state $s \in S$: $L : S \rightarrow D_1 \times \ldots \times D_n$, where $D_i$ represents the domain of a parameter $i \in [1,n]$ from a given state vector $\langle d_1 : D_1, \ldots, d_n : D_n\rangle$.
\end{itemize}


A valuation $\alpha \in V^C$  satisfies a formula $\phi$, written $\alpha \models \phi$, iff $\phi$ evaluates to TRUE
after replacing every variable $x \in \phi$ by $\alpha(x)$.

\begin{algorithm}
\caption{ClockRDL reachability algorithm}
\label{alg:reachability}
\begin{algorithmic}
\Function{CSReach}{$\mathcal{C}$, $\mathcal{A}$}
  \State $s_0 \gets \langle s_0^1 \ldots s_0^n \rangle$, where $s_0^a$ is the initial state of automaton $a \in \mathcal{A}$ 
  \State {\bf let} the {\bf set} of configurations $K\gets \{s_0\}$
  \State {\bf let} the {\bf queue} of configurations $Q\gets \{s_0\}$

  \While{$Q \neq \emptyset$}
    \State {\bf let} $s_i \gets first(Q)$
    \State $Q \gets Q \setminus \{s_i\}$

    \State $\{V^\mathcal{C}, F^T\} \gets clockValuations(\mathcal{C}, \mathcal{A}, s_i)$

    \ForAll {$v \in V^C$}
      \State $Actives \gets fireableWithClockValuation(F^T, v)$
      \ForAll {$toFire \in \left( \times_{i=1}^{i \in \left\vert{Actives}\right\vert} Actives_i \right)  $}
        \State $s_n \gets execute(toFire, s_i)$
        \If {$s_n \notin K$}
          \State $Q\gets Q \cup \{s_n\}$
          \State $K \gets K \cup \{s_n\}$
        \EndIf
      \EndFor
    \EndFor
  \EndWhile
\State\Return $K$
\EndFunction
\end{algorithmic}
\end{algorithm}

\paragraph{The function} $fireableWithClockValuation(F^T, v)$ selects for each automaton the transitions that can actually be fired given the current clocks valuation.

\paragraph{The function} $execute(toFire, s_i)$ executes all the transitions from the toFire set against the configuration $s_i$ producing the next configurations. Two cases are considered in this function: 
\begin{itemize}
\item If there is no shared state between the transitions they are executed in an arbitrary order producing only one next configuration;
\item If there is shared state between the transitions then all the possible interleavings are computed, potentially leading to the production of multiple next configurations.
\end{itemize}

\begin{algorithm}
\caption{Clock Valuations Function}
\label{alg:reachability}
\begin{algorithmic}
\Function{clockValuations}{$\mathcal{C}, \mathcal{A}, s_i$}
  \State {\bf let} the {\bf set} of valuations $V^\mathcal{C}\gets \emptyset$
  \ForAll {$a \in \mathcal{A}$}
    \State {\bf let} the {\bf set} of fireable transitions $F^T_a \gets \emptyset$
  \EndFor
  \State {\bf let} $\phi \gets true$

  \ForAll {$a \in A$}
    \State $F^T_a \gets fireableTransitions(a, s_i)$ \textcolor{gray}{-- \em fanout with true precondition}
    \State $\phi^a \gets formula4t(t_1, a) \lor \ldots \lor formula4t(t_n, a)$, forall $t_i \in F^T_a$
    \State $\phi \gets \phi \land \phi^a$
  \EndFor

\State\Return $\{allSat(\phi), F^T\}$
\EndFunction
\end{algorithmic}
\end{algorithm}


\begin{algorithm}
\caption{Boolean Function for a Transition}
\label{alg:reachability}
\begin{algorithmic}
\Function{beq4t}{$t, a$}
  \State {\bf let} the {\bf set} of stalled clocks $\mathcal{S} \gets alphabet(a) \setminus clocks(t)$
  \State $\phi \gets \lnot s_1 \land \ldots \land \lnot s_n $, forall $s_i \in \mathcal{S}$  \textcolor{gray}{-- \em block stalled clocks}

  \State $\phi^+ \gets c_1 \land \ldots \land c_n$, forall $c_i \in clocks(t)$ \textcolor{gray}{-- \em clocks do tick}
  \If { $\lnot isAlways(a)$ }
    \State $\phi^- \gets \lnot c_1 \land \ldots \land \lnot c_n$, forall $c_i \in clocks(t)$ \textcolor{gray}{-- \em clocks do not tick}
  \EndIf

\State\Return $\phi \land (\phi^+ \lor \phi^-)$
\EndFunction
\end{algorithmic}
\end{algorithm}


\section{CCSL Kernel Library}





\end{document}  